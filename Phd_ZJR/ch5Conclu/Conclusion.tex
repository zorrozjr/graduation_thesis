\newpage
\vskip 6mm
\pagestyle{myheadings}
\markboth{{\underline{\centerline{东南大学博士学位论文} }}}
{{\underline{\centerline{第七章 \ \ \ 总结与展望} }}}

\chapter{总结与展望} 
\setcounter{equation}{0}
\renewcommand{\theequation}{\thesection.\arabic{equation}}
 

\section{总结} 
本文主要在 Lyapunov 稳定性理论框架下,研究了具有脉冲控制的时滞复杂网络的动力学行为。
通过设计不同类型的脉冲控制,包括混杂脉冲控制,事件触发脉冲控制,分布式饱和脉冲控制等,利用时滞系统理论、脉冲控制理论以及 LMIs 等方法和工具,分别对具有不同类型时滞的复杂网络的全局动力学以及局部动力学进行了全面分析。所得结果完善了部分已有的结论,补充和丰富了脉冲控制器的设计方案以及时滞复杂网络动力学方面的理论成果,具体工作总结如下:

1. 研究了在混杂脉冲控制下耦合多稳态时滞忆阻神经网络的动态多同步以及静态多同步问题,
其中,混杂脉冲控制包括时滞脉冲控制和连续时间状态反馈控制。首先,基于状态空间划分方法和激励函数的几何特性,给出了系统具有多个局部指数稳定的周期轨道或平衡点的充分条件。其次, 利用新型的 Halanay 微分不等式和脉冲控制理论,分别得到了不依赖于时滞的动态多同步和静态多同步的相关判据。值得注意的是耦合多稳态时滞忆阻神经网络中状态时变时滞可微的限制和脉冲控制中的时变时滞的上界小于任意脉冲区间的要求都被取消,进一步降低了结果的保守性。

2. 讨论了事件触发时滞脉冲控制下具有耦合常时滞的复杂网络的全局指数同步问题。一种新型的基于二次 Lyapunov 函数的事件触发时滞脉冲控制机制被设计,此机制是将时滞脉冲控制考虑到事件触发机制中。其中,时滞脉冲控制仅在触发时刻被施加,而在非触发的时滞脉冲时刻不再发挥作用,进一步节约了网络资源。通过构造合适的辅助函数,并利用递推法、脉冲控制理论以及 Lyapunov–Razumikhin 技术,得到了基于 LMIs 的不依赖于时滞的同步判据并有效地避免芝诺现象。值得一提的是 Lyapunov 函数在触发的时间间隔内严格单调递减的要求被取消,改进了现有的结果。

3. 探讨了事件触发牵制脉冲控制下具有耦合比例时滞的基因振荡器网络的集群同步问题。首先,针对具有比例时滞的脉冲系统,设计了事件触发脉冲控制协议,保证了脉冲系统的全局渐近稳定性。在此基础上,进一步将牵制控制考虑到所建立的事件触发脉冲控制机制中,建立起事件触发牵制脉冲控制,研究了具有耦合比例时滞的基因振荡器网络的集群同步,得到了相应的集群同步判据并且消除了芝诺行为。不同于已有的事件触发脉冲控制,通过设计合适的算法,每个集群中只有一小部分节点在触发时刻被施加脉冲控制,进一步节约了网络资源。

4. 研究了分布式饱和脉冲控制下具有耦合时变时滞的鲁里叶网络的局部指数同步问题。利用反证法、脉冲系统的比较原理和平均脉冲区间的方法得到了基于 BMIs 的局部同步判据。为了降低保守性,选取最新的具有更多松弛变量的改进凸包表示法来处理分布式饱和脉冲项。目前,已有的关于饱和脉冲的结果基本上都是借助于收缩不变集来估计吸引域,这种方法要求 Lyapunov 函数在此集合中的任意非零点上是严格单调递减的。我们取消了这一限制,并开发了一种保守性更小的估计吸引域的方法。利用平均脉冲区间的方法处理脉冲信号,对脉冲间隔的上下界没有限制,这意味着允许存在非均匀分布的脉冲信号。

5. 分析了分布式饱和脉冲控制下具有切换拓扑的非线性时滞多智能体系统的局部一致性。利用反证法和 Lyapunov–Razumikhin 方法,得到了局部一致性判据。由于结构简单和计算方便,目前存在的处理饱和脉冲的结果大多数都是基于二次  Lyapunov 函数。然而,利用二次 Lyapunov 函数估计出的吸引域与真实的吸引域之间存在较大差距。为了降低由二次 Lyapunov 函数带来的保守性,构造了一个具有更多辅助矩阵的复合的依赖于脉冲时刻的 Lyapunov 函数,通过引入更多的决策变量来降低保守性,使得所估计的吸引域更接近真实吸引域。

\section{展望} 
时滞复杂网络具有广泛的应用背景,其动力学特性的研究是近年来的热门课题。 通过设计不同类型的脉冲控制器,本文对具有不同类型时滞的复杂网络的全局动力学以及局部动力学进行了分析,但仍有一些问题值得更深入的研究。
首先,对于混杂脉冲控制下时滞复杂网络的多同步研究已经取得了一些理论成果,但是其实际应用还需要进一步探索。
其次,对于事件触发脉冲控制,已经取得了一些成果,但尚有许多值得进一步探究的课题。
此外,目前对于饱和脉冲控制下时滞复杂网络的局部动力学研究相对较少,还存在着大量开放的问题。
基于本人的前期研究和本文的讨论,下面几个问题值得考虑:






%多同步的应用
1. 多同步的实际应用需要进一步探索。由于具有更高的存储能力,耦合多稳态时滞神经网络在处理高维数据时具有更高的效率。目前耦合多稳态时滞神经网络的多同步大多还是在理论研究上,只有较少的文献将耦合多稳态时滞神经网络的多同步应用于图像处理中,今后可以考虑进一步探索其在联想记忆、模式识别、人工智能等相关领域的应用。此外,本文仅讨论了耦合多稳态时滞忆阻神经网络模型,更多的类型的耦合多稳态神经网络值得进一步探索,以处理不同的实际问题。


% 分布式的事件触发策略
2. 事件触发脉冲控制值得进一步改进。到目前为止,关于事件触发脉冲控制的研究成果,大部分都是同步事件触发机制和集中式控制策略。首先,同步事件触发机制
需要网络中所有节点在同一时刻的采样信息来判断触发条件是否满足,这可能导致不必要的资源浪费。因此,在设计事件触发脉冲控制时,每个节点都有单独的触发时刻进行信息传输和控制更新的异步事件触发机制值得被考虑。此外,相比于集中式的脉冲控制,分布式的脉冲控制中每个节点可以以协作的方式低代价地完成控制任务,因此,分布式脉冲控制在设计事件触发脉冲控制机制时值得被充分考虑。 


%%%%传感器饱和、速率饱和
3. 关于饱和脉冲控制有待进一步分析。 对于执行器饱和而言,传输信号速率约束比幅值大小的约束对系统的影响更为显著,因此,具有执行器速率饱和的脉冲控制具有重要的研究意义。此外,目前关于饱和脉冲控制的研究成果大部分考虑的是执行器饱和的情况。而在实际的工业生产中,由于制造技术以及使用安全等诸多因素限制,传感器会无法识别或者提供幅值过大的信号,因而产生传感器的饱和特性。因此,具有传感器饱和的脉冲控制也值得进一步研究。
 
%网络问题((随机)丢包、dos攻击)
4. 多种网络问题值得进一步考虑。本文的结果都是在理想的网络环境中, 而由于通信噪声、带宽受限、信息拥塞等原因,实际无线通信网络的丢包普遍存在。网络丢包会降低系统的性能甚至导致不稳定,因此研究丢包环境下脉冲控制是很有必要的。此外, 由于通信环境的不稳定,复杂网络会
受到很多干扰,例如网络攻击。网络攻击会破坏节点之间的通信链路,从而破坏系统的性能。 因此,具有网络攻击的复杂网络的信息安全问题也具有重要的研究价值。

%网络拓扑
5. 复杂网络的网络拓扑结构值得进一步研究。 为了实现所期待的动力学行为,本文考虑的网络拓扑都是连通或强连通的。而在许多实际网络中,网络拓扑可能是切换的,但每个切换的子图不满足连通或强连通条件。目前,已经有一些关于连续控制下
具有联合连通或序列联通的复杂网络同步的研究成果,而对不连续的脉冲控制来说,不连通的复杂网络的研究还较少。 脉冲控制是否能在联合连通或序列联通的网络拓扑下实现复杂网络的同步,这值得进一步探索。
 