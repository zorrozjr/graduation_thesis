\newpage
\appendix
\chapter{博士期间撰写和发表的论文}

\markboth{\underline{\centerline{东南大学博士学位论文}}}
 {\underline{\centerline{博士期间撰写和发表的论文、专利}} }

\vskip 20pt \noindent 一、发表的期刊论文列表:

\vspace*{3pt} [1]. 第一作者. Local synchronization of directed Lur'e
networks with coupling delay via distributed impulsive control subject to actuator
saturation, \textit{\textbf{IEEE Transactions on Neural Networks and Learning Systems}}, 2022, DOI: 10.1109/TNNLS.2021.3138997. ({\bf SCI} 收录)

\vspace*{3pt} [2].  第一作者.
Synchronization analysis for complex dynamical networks with coupling delay via
event-triggered delayed impulsive control, \textit{\textbf{IEEE Transactions on Cybernetics}},
51(11):5269--5278, 2021. ({\bf SCI} 收录)


\vspace*{3pt} [3].  第一作者. Dynamical and static multisynchronization
analysis for coupled multistable memristive neural networks with hybrid control, \textbf{\textit{Neural
Networks}}, 143:515-524, 2021. ({\bf SCI} 收录)

\vspace*{3pt} [4].  第一作者. Dynamical and static multisynchronization of
coupled multistable neural networks via impulsive control, \textit{\textbf{IEEE Transactions on Neural
        Networks and Learning Systems}}, 29(12):6062--6072, 2018. ({\bf SCI} 收录)

\vspace*{3pt} [5].  第四作者. Extended dissipativity performance
of high-speed train including actuator faults and probabilistic time-delays under resilient
reliable control, \textbf{\textit{IEEE Transactions on Systems, Man, and Cybernetics: Systems}}, 51(6):
3808-3819, 2021. ({\bf SCI} 收录)

\vspace*{3pt} [6]. 第四作者. Enhanced $L_2$--$L_\infty$ state
estimation design for delayed neural networks including leakage term via quadratic-type
generalized free-matrix-based integral inequality, \textbf{\textit{Journal of the Franklin Institute}},
356(13):7371-7392, 2019. ({\bf SCI} 收录)




\clearpage
\vskip 20pt
\noindent  二、在审的期刊论文列表:

\vspace*{3pt} [1].  第一作者. Distributed saturated impulsive
control for local consensus of nonlinear time-delay multi-agent systems with switching
topologies, \textbf{\textit{IEEE Transactions on Automatic Control}}. Under review.


\vspace*{3pt} [2].  第一作者. Event-triggered pinning impulsive
control for cluster synchronization of coupled genetic oscillator networks with
proportional delay, \textit{\textbf{IEEE Transactions on Systems, Man, and Cybernetics: Systems}}.  Under review.

\vskip 20pt \noindent 三、发表的会议论文列表:

\vspace*{3pt} [1]. 第一作者. Finite-time average consensus of multi-agent
systems with impulsive perturbations. \textbf{\textit{2020 12th International Conference on
Advanced Computational Intelligence}},  (ICACI 2020), 558--563, August 14--16, 2020, Dali, China. (EI 收录)

\vspace*{3pt} [2].  第一作者. Impulsive local leader-following consensus and estimation of
domain of attraction of multi-agent systems with actuator saturation.\textit{ \textbf{2021 11th
International Conference on Information Science and Technology}}, (ICIST 2021), 553--558, May 21--23, 2021, Chengdu, China. (EI 收录)


\chapter{博士期间主持和参加的科研项目、学术会议和获得的荣誉}

\markboth{\underline{\centerline{东南大学博士学位论文}}}
 {\underline{\centerline{博士期间主持和参加的科研项目、学术会议和获得的荣誉}} }

\vskip 15pt \noindent 一、博士期间主持和参加的科研项目:

\vspace*{3pt}\indent1. 不连续控制下多稳态系统的动力学分析,东南大学优秀博士学位论文培育基金,(No. YBPY1967),主持

%\vspace*{3pt}\indent
\vspace*{3pt}\indent2. 基于群体智能的分布式优化理论、方法及应用研究,国家自然科学基金重点项目~(No. 61833005),参与

\vspace*{3pt}\indent3. 多智能体最优合作调控及其在电力系统中应用,国家自然科学基金面上项目~(No. 61573096),参与


\vskip 15pt \noindent 二、博士期间参加的学术会议与学术论坛:

\vspace*{3pt} 1.\, 2021~年~05~月,第 11 届信息科学与技术国际会议 (ICIST 2021),成都,四川

\vspace*{3pt} 2.\, 2020~年~11~月,第 44 届复杂系统与网络科学研究中心论坛, 南京,江苏

\vspace*{3pt} 3.\, 2020~年~08~月,第 12 届高级计算智能国际会议 (ICACI 2020), 大理,云南

\vspace*{3pt} 4.\, 2019~年~10~月,第 12 届复杂系统与应用国际研讨会, 南京,江苏 

\vspace*{3pt} 5.\, 2019~年~07~月,第 38 届中国控制会议 (CCC 2019),广州,广东

\vspace*{3pt} 6.\, 2018~年~10~月,2018 年国际复杂系统与网络论坛  (IWCSN 2018),南京,江苏



\vskip 15pt \noindent 三、博士期间科研工作经历:

\indent1.\, 
2019.10--2019.11 香港城市大学~(CityU),香港,中国 (助理研究员)。

\vskip 15pt \noindent 四、博士期间获得的荣誉:

\vspace*{3pt} 1.\, 2018~年~12~月,东南大学新生奖学金



